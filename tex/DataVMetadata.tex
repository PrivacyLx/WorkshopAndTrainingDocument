One of the things that laypeople have a hard time understanding is the difference between metadata and data. Though many people have heard the term metadata before (especially since the Snowden Revelations), they seem to misunderstand how much information one can glean from metadata. In this section we will describe different methods of teaching about metadata, and teaching the dangers of how metadata collection can be used to infer information about their lives.

\subsection{Definition of Data and metadata}

The Merriam-Webster dictionary defines metadata as data that provides information about other data. An example of metadata is the addresses written on an envelope, or the phone numbers involved in the phone call. Though this data doesn't tell what the contents of the envelope or the phone call are, they do describe who is having a conversation.

However, most laypeople do not know how the Internet works, so they do not understand how this analogy translates to the Internet. Perhaps the easiest metadata on the Internet to understand is e-mail metadata, since much of the metadata in e-mail closely resembles the metadata in the physical equivalent. However, there is a lot more meta-data that exists in e-mails that don't exist in the physical equivalent, such as an extremely accurate time-stamp, mail client information, and others.

After discussing how e-mail amplifies metadata compared to the physical world, depending on the audience it may be appropriate to discuss what metadata is sent to which parties when one browses the Internet. Many individuals don't understand that their browsing history is accessible to many different companies based on the IP addresses that arrive at random routers on the Internet. Discussing this with them may help them understand how the very act of browsing takes data from them without their consent.

\subsection{Dangers of Metadata}

After discussing this, many laypeople may assert that this is simply not that critical. After all, if they can only tell who one is talking to, privacy is still preserved, since the contents of the message remain unreadable to third parties (in some cases). This can only be countered by discussing the predictive value of metadata. Similar to discussing what metadata is, it can help to start with an example that people are already familiar with, such as phone calls. On example that has been used to great success is the STI clinic example.

Imagine someone is looking at your phone records. They see a call from an STI clinic to your phone, which lasts 5 minutes. They then see a call from your phone to your SO's phone, which lasts 30 minutes. They then see a call between your phone and a doctor. Is it still OK, since they did not hear the contents of the calls?

This one example may be extreme, but a 2016 study showed that they could detect many different things from such metadata, including political and religious affiliations, romantic relationships, and, in one case, even a health problem like cardiac arrhythmia \cite{phonemetadata}. Indeed, call records can reveal a lot of information about regular people. This is only considering phone records. Imagine what could be predicted about individuals based on their Web browsing history, by any company that owns a router that your packs happen to go through. Imagine what Facebook can predict about you with the data you provide them.

A few laypeople may say that companies like Facebook wouldn't predict things like that. However, it is easy to point out that they are financially incentivized to do so. The more Facebook can predict about you, the more money they can get from advertisers who wish to target certain populations. Therefore, Facebook and companies like it are incentivized to ensure that you have as little privacy as possible. Indeed, they are incentivized to try to predict your sexuality, which they can do with at least 88\% accuracy based on nothing but the pages you like \cite{facebookhomosexuality}.

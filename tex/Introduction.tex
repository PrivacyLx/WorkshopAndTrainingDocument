This document is a (hopefully) living curriculum for teaching cryptoparties and otherwise informing members of the the general public (also referred to as a layman) about CyberSecurity principles, issues, and tools. It focuses on tools that enhance digital privacy, anonymity, and anti-censorship, rather than technologies that help in enterprise security (such as Intrusion Detection Systems, etc.).

\subsection{Organization}
This document has eight sections. This section (Sec. \ref{sec:Introduction}) introduces the content of the document, the origins of the document, and what the document aims to achieve. The Motivation section (Sec. \ref{sec:Motivation}) includes a description of potential motivations one may have for persuing CyberSecurity information. It can be used to understand one's audience better, and contains arguments for why laymen should care about CyberSecurity and privacy. In addition, it counters arguments such as the ``nothing to hide'' argument. The Basic Principles of Security section (Sec. \ref{sec:Principles}) provides information about the fundamentals of security, such as a definition of security, and creates the foundation that laymen will need to discuss different elements of security. The Data vs. Metadata section (Sec. \ref{sec:DataVMetadata}) then discusses the difference between the data we humans observe and the data about the data we observe, called metadata. It then describes how metadata can be used to violate our privacy, and cites examples. The next section (Sec. \ref{sec:ThreatModeling}) discusses Threat Modeling, and gives the layman the tools to communicate which threats and adversaries he or she is worried about. The section that follows (Sec. \ref{sec:Encryption}) discusses encryption, its multiple forms, and its limits of operation.The final content section (Sec. \ref{sec:Tools}) discusses various privacy, anonymity, and anti-censorship tools. Finally, the last section (Sec. \ref{sec:Conclusion}) concludes.

\subsection{Need for this Document}

There exists a wealth of information about CyberSecurity and Privacy on the Internet. However, this information is often too technical to be used to teach laymen, or exists scattered in different places, making it hard to form an encompassing curriculum. By centralizing this information, we hope to make it easy for trainers to find the necessary information to start a cryptoparty or other training event with ease. In addition, we hope to keep this a living, up-to-date document, allowing for dissemination of accurate, helpful information.

\subsection{Intended Audience}

The intended audience of this document is a cybersecurity and/or privacy trainer, or cryptoparty host. This document provides information on the concepts and technologies to teach, and why trainers should teach these technologies. We assume that the trainers have a technical background, and can understand these principles and technologies.

We assume that the trainer will be hosting events in which he will teach the layman about CyberSecurity and privacy. Since the audience of this cirriculum is the layman, the trainer should assume no knowledge about computer systems or networks beyond basic computer operation on common Operating Systems such as MacOS or Windows. The trainer should assume no knowledge of the command line or specialized technologies. This curriculum should be aimed at the lowest common denominator, and it is up to the trainer to adapt should his or her audience be more technical than anticipated. Because of this, this document will provide descriptions and examples that the layman will (hopefully) be able to understand.

Though the original intended audience for the developed curriculum was the Portuguese public, we believe this document provides information that is general enough to be useful to members outside of this population.

\subsection{About the Authors}

Kevin Gallagher is a Ph.D. candidate at the NYU Tandon School of Engineering. He belongs to the Center for CyberSecurity, and focuses his research on usable security, privacy, and anonymity. His thesis is about the Usability of Tor and the Tor Browser. In his spare time he works to organize the Tor New York City meet-ups, teaches cryptoparties, plays Dungeons and Dragons, and sings.
